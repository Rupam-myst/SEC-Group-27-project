\documentclass[a4paper,12pt]{article}
\usepackage{graphicx}
\usepackage{titling}
\usepackage{geometry}
\usepackage{setspace}
\usepackage{parskip}
\usepackage{hyperref}
\usepackage{tocloft}
\usepackage[utf8]{inputenc}
\usepackage{amsmath}
\geometry{a4paper, margin=1in}

% Title for the cover page
\title{\Huge \textbf{Lab Notebook}\\[1.5cm]}
\author{}
\date{}

\begin{document}

% Cover Page
\begin{titlingpage}
    \centering
    \vspace*{4cm}
    {\Huge \bfseries Lab Notebook}\\[1.5cm]
    
    % Placeholder for logo/image
    \includegraphics[width=0.2\textwidth]{example-image} \\[1cm]
    
    {\Large \textbf{Prepared by:}} \\[1cm]
    
    % Names and details in tabular format for better alignment
    \begin{tabular}{lll}
        \textbf{Name:} & \textbf{Rupam Mallick} & \textbf{Roll No.:} 30085323014 \\
        \textbf{Dept:} & BSC IT (CS) & \\
        \\
        \textbf{Name:} & \textbf{Sohani Mondal} & \textbf{Roll No.:} 30085323004 \\
        \textbf{Dept:} & BSC IT (CS) & \\
        \\
        \textbf{Name:} & \textbf{Arijit Podder} & \textbf{Roll No.:} 30054623001 \\
        \textbf{Dept:} & BSC IT (AI) & \\
        \\
        \textbf{Name:} & \textbf{Shuvajit Mondal} & \textbf{Roll No.:} 30059223005 \\
        \textbf{Dept:} & Forensic Science and Technology & \\
        \\
        \textbf{Name:} & \textbf{Ayon Halder} & \textbf{Roll No.:} 30084323017 \\
        \textbf{Dept:} & BSC IT Data Science &
    \end{tabular}
    
    \vfill
    {\Large \textbf{Date:}} \\[0.5cm]
    {\large \today}
\end{titlingpage}

\newpage

% Index page
\section*{Index}
\tableofcontents
\newpage

% Lab Notebook Entries
\section{Lab Notebook Entries}

\subsection{Entry 1: Rupam Mallick}
\textbf{Department: BSC IT (CS)} \\
\textbf{Roll No.: 30085323014} \\

\textbf{Task:} \\
Front page and index page creation.

\textbf{Details:} \\
Write the details about what was done for the front page and index creation, including the steps followed, software used, and any relevant notes.

\subsection{Entry 2: Sohani Mondal}
\textbf{Department: BSC IT (CS)} \\
\textbf{Roll No.: 30085323004} \\

\textbf{Task:} \\
Create a local repository. Build a C program of a calculator in the local repository, commit, and publish it as a public repository.

\textbf{Details:} \\
Here you will describe the process of creating the repository, writing the C code for the calculator, and steps involved in publishing the repository.

\subsection{Entry 3: Arijit Podder}
\textbf{Department: BSC IT (AI)} \\
\textbf{Roll No.: 30054623001} \\

\textbf{Task:} \\
Edit a mind reader application.

\textbf{Details:} \\

\title{Mind Reader Application - Lab Assignment}
\author{Name : Arijit Podder}
\date{Semester : 2\\
Department : Information Science \\
Course : B.Sc in IT (Artificial Intelligence) \\
Subject : Software Tools and Technology \\
Session : 2023-24 \\
Roll No : 30054623001 \\
Registration No : 233002410553}

\maketitle

\section*{Task}
Your professor created a mind reader application and wants you to try it out. After running the program, you found the submit button looks dull. You renamed it "Chin Tapak Dum Dum," but the button became disproportionate. Your task is to fix the button issue and create a pull request with the solution.

\section*{Procedure}

\subsection*{1. Clone the Repository}
\begin{itemize}
    \item I used GitHub Desktop to clone the repository:\\ 
    \url{https://github.com/GeekAyan/STT}
    \item Then I opened GitHub Desktop, clicked on "File" then "Clone Repository", pasted the URL, and selected my local directory.
    \item Navigate to the project directory.
\end{itemize}

\subsection*{2. Run the Application}
\begin{itemize}
    \item I opened the project in Visual Studio Code.
    \item Then I followed the detailed run instructions provided in the \texttt{README.md} file to set up any necessary dependencies and configurations.
    \item I ran the application to ensure everything is working as expected.
    \item Observe the application's user interface, particularly the submit button.
\end{itemize}

\subsection*{3. Identify and Rename the Button}
\begin{itemize}
    \item Locate the submit button code in the application’s source files.
    \item Rename the button text to "Chin Tapak Dum Dum."
    \item Notice that the button has become disproportionate due to the increased text length.
\end{itemize}

\subsection*{4. Fix the Button Size}
\begin{itemize}
    \item Analyze the layout code that controls the button's appearance.
    \item Adjust the width and height properties or use appropriate CSS/JavaFX adjustments to make the button proportionate.
    \item Test the application to ensure the button now displays correctly and does not affect other UI elements.
\end{itemize}

\subsection*{5. Commit and Push the Changes}
\begin{itemize}
    \item Commit the changes with a descriptive message: \texttt{git commit -m "Improve Button Proportions and Renamed the Submit Button"}.
    \item Push the changes to your forked repository on GitHub.
\end{itemize}

\subsection*{6. Create a Pull Request}
\begin{itemize}
    \item I went to the original GitHub repository on my web browser.
    \item Clicked on "Pull Requests" then "New Pull Request."
    \item Compare my branch with the main branch of the original repository.
    \item Added a title and description explaining my changes and why they were made.
    \item Submit the pull request to the original repository.
\end{itemize}

\subsection*{7. Review and Merge}
\begin{itemize}
    \item Wait for the repository owner to review my pull request.
    \item If accepted, the changes will be merged into the main project.
\end{itemize}

\newpage

\section*{Symbol Mind Reading Java Application}
\begin{verbatim}
import java.awt.*;
import java.awt.event.*;
import java.util.Random;

public class SymbolApp extends Frame implements ActionListener {

    private Label[] symbolLabels = new Label[99];
    private Button submitButton;
    private String specialSymbol;
    private String selectedSymbol;

    public SymbolApp() {
        // Generate a random special symbol
        Random rand = new Random();
        specialSymbol = Character.toString((char) (rand.nextInt(94) + 33)); 
        selectedSymbol = specialSymbol;

        // Setting up the main frame
        setLayout(new BorderLayout());
        setSize(800, 700);
        setTitle("Symbol App");

        // Adding instruction message
        TextArea instruction = new TextArea(
            "Think of any two digit number. Now reverse it and find the difference 
            of them.\n" +
            "Now find the number you got and remember the symbol from the panel 
            below.\n" +
            "Don't tell me, I'll read your mind! Hit the below button when you 
            are ready to see the magic!",
            5, 60, TextArea.SCROLLBARS_NONE
        );
        instruction.setEditable(false);
        instruction.setFont(new Font("Arial", Font.PLAIN, 16));
        add(instruction, BorderLayout.NORTH);

        // Panel for symbols
        Panel symbolPanel = new Panel(new GridLayout(11, 9));
        for (int i = 0; i < 99; i++) {
            String symbol = (i % 9 == 0) ? specialSymbol : Character.toString((char) 
            (33 + (i % 94)));
            symbolLabels[i] = new Label(i + ": " + symbol); // Numbering symbols
            symbolLabels[i].setAlignment(Label.CENTER);
            symbolPanel.add(symbolLabels[i]);
        }
        add(symbolPanel, BorderLayout.CENTER);

        // Panel for submit button
        Panel controlPanel = new Panel(new FlowLayout());
        submitButton = new Button("Chin Tapak Dum Dum");
        submitButton.setFont(new Font("Times New Roman", Font.BOLD, 15));
        submitButton.setPreferredSize(new Dimension(200, 60));
        submitButton.addActionListener(this);
        controlPanel.add(submitButton);
        add(controlPanel, BorderLayout.SOUTH);

        // Setting up the window close event
        addWindowListener(new WindowAdapter() {
            public void windowClosing(WindowEvent we) {
                System.exit(0);
            }
        });

        setVisible(true);
    }

    public void actionPerformed(ActionEvent ae) {
        // Clear the current content and display the selected symbol
        if (ae.getSource() == submitButton) {
            removeAll();
            setLayout(new BorderLayout());
            Label resultLabel = new Label(selectedSymbol, Label.CENTER);
            resultLabel.setFont(new Font("Arial", Font.BOLD, 50));
            add(resultLabel, BorderLayout.CENTER);
            validate();
            repaint();
        }
    }

    public static void main(String[] args) {
        new SymbolApp();
    }
}
\end{verbatim}
\newpage

\subsection{Entry 4: Shuvajit Mondal}
\textbf{Department: Forensic Science and Technology} \\
\textbf{Roll No.: 30059223005} \\

\textbf{Task: \documentclass{article}

\usepackage{amsmath} % For advanced math typesetting
\usepackage{graphicx} % For including images

\title{Introduction to \LaTeX}
\author{Shuvajit Mondal}
\date{\today}

\begin{document}

\maketitle

\begin{abstract}
This document serves as a brief introduction to \LaTeX, a widely-used typesetting system. It covers the basics of document structure, formatting, mathematical typesetting, and features like cross-referencing, tables, and figure handling.
\end{abstract}

\section{What is \LaTeX?}
\LaTeX{} is a high-quality typesetting system that is widely used for producing scientific and technical documents. It is not a word processor; instead, \LaTeX{} allows authors to focus on content rather than design, with an emphasis on logical structure. Developed in the 1980s by Leslie Lamport, \LaTeX{} builds on Donald Knuth’s \TeX{}, adding macros to simplify document production.

A few key features of \LaTeX{} include:
\begin{itemize}
    \item Automatic generation of bibliographies, indexes, and tables of contents.
    \item Handling complex mathematical notations.
    \item Consistent and professional typography.
    \item Easy cross-referencing of sections, equations, and figures.
\end{itemize}

\section{Document Structure}
Every \LaTeX{} document begins with a document class declaration, which determines the overall layout. Some common classes are \texttt{article}, \texttt{report}, and \texttt{book}. The document itself is enclosed between \texttt{\textbackslash begin\{document\}} and \texttt{\textbackslash end\{document\}}.

Here’s an example of a very simple \LaTeX{} document:

\begin{verbatim}
\documentclass{article}

\begin{document}

\title{Sample Document}

\author{Author Name}

\date{\today}

\maketitle

\section{Introduction}

This is the introduction to our document.

\end{document}
\end{verbatim}

You can add sections, subsections, and paragraphs to structure your document logically. \LaTeX{} automatically numbers sections and subsections for you.

\subsection{The Preamble}
The preamble is the part of the document before \texttt{\textbackslash begin\{document\}}. Here, you can load packages to extend \LaTeX's functionality, set global formatting options, and define custom commands.

Example of including a package:

\texttt{\textbackslash usepackage\{amsmath\}} % For advanced math typesetting

\section{Mathematical Typesetting}
\LaTeX{} excels at typesetting mathematical equations, both inline and as standalone expressions. Inline math is written within \$ symbols, like this: $a^2 + b^2 = c^2$. For larger expressions, the \texttt{equation} environment is used.

For example, here’s the quadratic formula:
\[
x = \frac{-b \pm \sqrt{b^2 - 4ac}}{2a}
\]

\subsection{Dotted Expressions}
Dots are often used to indicate that a pattern continues. You can use different types of dots for various purposes in mathematical typesetting:

\[
1 + 2 + 3 + \cdots + n = \frac{n(n+1)}{2}
\]

\[
A = \begin{pmatrix}
a_{11} & a_{12} & \cdots & a_{1n} \\
a_{21} & a_{22} & \cdots & a_{2n} \\
\vdots & \vdots & \ddots & \vdots \\
a_{m1} & a_{m2} & \cdots & a_{mn}
\end{pmatrix}
\]
In the matrix example, \texttt{\textbackslash vdots} represents vertical dots, \texttt{\textbackslash ddots} represent diagonal dots, and \texttt{\textbackslash cdots} represents horizontal dots.

\section{Inserting Figures and Tables}
Including images and tables is simple in \LaTeX. For figures, you use the \texttt{figure} environment and the \texttt{\textbackslash includegraphics} command.

Here is an example of how to insert a figure:

\begin{figure}[h]
    \centering
    \includegraphics[width=0.5\linewidth]{image1.png}
    \caption{Enter Caption}
    \label{fig:enter-label}
\end{figure}

Tables are created using the \texttt{table} and \texttt{tabular} environments. For example:

\begin{table}[h]
    \centering
    \begin{tabular}{|c|c|c|}
        \hline
        Column 1 & Column 2 & Column 3 \\
        \hline
        1 & 2 & 3 \\
        4 & 5 & 6 \\
        \hline
    \end{tabular}
    \caption{A simple table.}
    \label{tab:example}
\end{table}

\section{Cross-Referencing and Citations}
\LaTeX{} provides robust tools for cross-referencing sections, equations, and figures. For example, to reference Figure~\ref{fig:enter-label}, use the \texttt{\textbackslash ref} command. You can also automatically generate a bibliography by using BibTeX or BibLaTeX.

\section{Compiling the Document}
To see the formatted output of your \LaTeX{} document, you need to compile it. The most common compiler is \texttt{pdflatex}, which generates a PDF from your \LaTeX{} source code. Online editors like Overleaf provide a user-friendly interface for compiling and editing \LaTeX{} documents.

\section{Conclusion}
This brief introduction to \LaTeX{} covers its fundamental concepts and features. As you become more comfortable, you can explore advanced topics such as custom commands, style files, and the extensive array of packages available for \LaTeX. It is a powerful tool that ensures professional, consistent document formatting, making it a favorite for researchers, scientists, and academic writers.

\end{document}
 } \\
Introductiion to Latex Assignment

\textbf{Details:} \\
Basics of document structure, formatting, mathematical typesetting, and features

\subsection{Entry 5: Ayon Halder}
\textbf{Department: BSC IT Data Science} \\
\textbf{Roll No.: 30084323017} \\

\textbf{Task:} \\
Demonstrate proficiency in Git branching, merging, and conflict resolution.

\textbf{Details:} \\
Describe the steps followed to create branches, merge them, and handle conflicts. Include any code snippets or Git commands used, along with explanations.





\end{document}